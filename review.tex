%!Tex Program = xelatex
%\documentclass[a4paper]{article}
\documentclass[a4paper]{ctexart}
\usepackage{xltxtra}
%\setmainfont[Mapping=tex-text]{AR PL UMing CN:style=Light}
%\setmainfont[Mapping=tex-text]{AR PL UKai CN:style=Book}
%\setmainfont[Mapping=tex-text]{WenQuanYi Zen Hei:style=Regular}
%\setmainfont[Mapping=tex-text]{WenQuanYi Zen Hei Sharp:style=Regular}
%\setmainfont[Mapping=tex-text]{AR PL KaitiM GB:style=Regular} 
%\setmainfont[Mapping=tex-text]{AR PL SungtiL GB:style=Regular} 
%\setmainfont[Mapping=tex-text]{WenQuanYi Zen Hei Mono:style=Regular} 

\usepackage{listings}
\usepackage{xcolor}
\usepackage{amsmath}
\usepackage{amsthm}
\usepackage{amssymb}
\usepackage{mathrsfs}
\usepackage{enumitem}  
\usepackage{tikz}

\definecolor{codegreen}{rgb}{0,0.6,0}
\definecolor{codegray}{rgb}{0.5,0.5,0.5}
\definecolor{codepurple}{rgb}{0.58,0,0.82}
\definecolor{backcolour}{rgb}{0.95,0.95,0.92}

\lstdefinestyle{mystyle}{
    backgroundcolor=\color{backcolour},   
    commentstyle=\color{codegreen},
    keywordstyle=\color{magenta},
    numberstyle=\tiny\color{codegray},
    stringstyle=\color{codepurple},
    basicstyle=\ttfamily\footnotesize,
    breakatwhitespace=false,         
    breaklines=true,                 
    captionpos=b,                    
    keepspaces=true,                 
    numbers=left,                    
    numbersep=5pt,                  
    showspaces=false,                
    showstringspaces=false,
    showtabs=false,                  
    tabsize=2
}

\lstset{style=mystyle}

\title{数值分析复习要点}
\author{王何宇}
\date{}

\newcommand{\remark}[1]
{\noindent {\bf Remark {#1}}}

\newcommand{\definition}[1]
{\noindent {\bf Definition {#1}}}

\newcommand{\hl}[1]
{\noindent {\bf {#1}}}

\newtheorem{theorem}{定理}[section]

\begin{document}
\maketitle
\pagestyle{empty}

\section{Finite precision arithmetic and error}

\begin{itemize}
  \item 计算机使用二进制系统存储数字(64 位和 32 位)
  \begin{itemize}
    \item 64 位:符号位 $s$(1 位),指数位 $c$(11 位),尾数位 $f$(52 位), 十进制表示为:
    $$ x =  (-1)^s \times 2^{c - 1023} \times (1.f) $$
    \item 32 位:符号位 $s$(1 位),指数位 $c$(8 位),尾数位 $f$(23 位), 十进制表示为:
    $$ x =  (-1)^s \times 2^{c - 127} \times (1.f) $$
    \item 要会转换,转换时别忘了 $1.f$ 中的 $1$ 和指数偏移量(64 位的 $1023$,32 位的 $127$).
    \item 机器精度(Machine Epsilon): $\epsilon_{mach} = 2^{-52}$(64 位),$\epsilon_{mach} = 2^{-23}$(32 位).
    \item 最小正规化数(Smallest Normalized Number): $x_{min} = 2^{-1022}$(64 位),$x_{min} = 2^{-126}$(32 位).
    \item 最大数(Largest Number): $x_{max} = (2 - 2^{-52}) \times 2^{1023}$(64 位),$x_{max} = (2 - 2^{-23}) \times 2^{127}$(32 位).
  \end{itemize}
  \item 舍入误差(Rounding Errors)
  \begin{itemize}
    \item 绝对误差(Absolute Error): $e = x - \tilde{x}$.
    \item 相对误差(Relative Error): $e_r = \left|\frac{x - \tilde{x}}{x}\right|$.
  \end{itemize}
  \item 减少舍入误差的方法
  \begin{itemize}
    \item 避免相近的数相减.
    \item 重组计算顺序,比如累加时先加小数后加大数.
    \item 尽量降低运算次数.
  \end{itemize}
\end{itemize}

\section{Linear systems of equations}

\begin{itemize}
  \item 高斯消去法(Gaussian Elimination)
  \begin{itemize}
    \item 计算复杂度为 $O(n^3)$.
    \item 部分选主元:在每一步消元前,都从当前列选出绝对值最大的元素作为主元并交换行。
    \item 比例选主元。考虑行内元素的数量级差异。先计算每行的最大元素作为“比例因子”,再选择“当前列元素/比例因子”绝对值最大的行作为主元行。
    \item 选主元的目的是减少舍入误差,同时避免因主元为零或接近零而导致的计算失败。
  \end{itemize}
  \item LU 分解(LU Decomposition)
  \begin{itemize}
    \item 每一次 Gauss 消去相当于左乘一个初等变换阵,每一次选主元的行交换相当于左乘一个置换矩阵. 将二者记录下来,
    最终可得到 $PA = LU$,其中 $P$ 是置换矩阵,$L$ 是单位下三角矩阵,$U$ 是上三角矩阵.
    \item 可以用于多次求解同一系数矩阵的线性方程组. 对于变化的常数项 $b$,只需重新进行前后代回代即可.
    \item 务必确认能计算整个过程,包括前代和回代.
    \item 会利用矩阵性质解决 $A^{-1}$,$A^T x = b$ 等相关问题。
  \end{itemize}
  \item 相关的线性代数知识
  \begin{itemize}
    \item 线性方程组 $Ax = b$ 有唯一解的充分必要条件是 $A$ 可逆($\det(A) \neq 0$).
    \item 交换阵是正交阵, $P^TP = I$. Gauss 消去时的初等变换阵的逆是对应下三角元素取负.
    \item $(AB)^T = B^T A^T$, $\det(P) = \pm 1$ (取决于奇交换还是偶交换)
  \end{itemize}
\end{itemize}

\section{Root finding}

\begin{itemize}
  \item 二分法(Bisection Method)
  \begin{itemize}
    \item 注意前提条件:适用于连续函数 $f(x)$ 在区间 $[a,b]$ 上有根的情况,要求 $f(a)f(b) < 0$. 相关分析使用介值定理,收敛性可用区间套。
    \item 收敛速度慢(线性),仅为 1 阶收敛.
    \item 误差估计:第 $n$ 步后,误差上限为 $\frac{|b-a|}{2^n}$. (一步迭代误差下降一半)
  \end{itemize}
  \begin{itemize}
    \item 不动点迭代法(Fixed Point Iteration):将 $f(x) = 0$ 改写成等价形式 $x = g(x)$,然后迭代 $x_{n+1} = g(x_n)$. 
    \item 不动点存在性:若 $g(x)$ 在 $[a, b]$ 连续,且值域包含于 $[a, b]$,则 $[a,b]$ 内至少存在一个不动点。
    \item 唯一性与收敛:若还有 $|g'(x)| \leq k < 1$ 对所有 $x \in (a,b)$ 成立,则不动点唯一,且对任意初值 $x_0 \in [a,b]$,迭代都收敛。
    \item 收敛速度:注意这是一个局部性质,需假设迭代序列收敛到某个不动点 $x^*$,然后根据 $g'(x^*)$ 的值判断收敛阶。它只在 $x^*$ 附近成立。
    \begin{theorem}
    若不动点迭代 \( x_{n+1} = g(x_n) \) 收敛于 \( x^* \in [a, b] \),且 \( g \) 满足 \( g \in C^p[a, b] \) 以及  
\[
0 = g'(x^*) = g''(x^*) = \ldots = g^{(p-1)}(x^*), \quad g^{(p)}(x^*) \neq 0
\]  
(即在 \( x = x^* \) 处前 \( p - 1 \) 阶导数为零),则收敛阶为 \( p \),且渐近误差常数为  
\[
\lambda = \frac{|g^{(p)}(x^*)|}{p!}.
\]
    \end{theorem}
  \end{itemize}
  \item 牛顿-拉弗森法(Newton-Raphson Method)
  \begin{itemize}
    \item 迭代公式:$x_{n+1} = x_n - \frac{f(x_n)}{f'(x_n)}$.
    \item 收敛性:局部收敛,要求初值足够接近真根 $x^*$. 对于单根($f(x^*)=0, f'(x^*) \neq 0$),具有二阶收敛速度. 对于重根($f(x^*)=f'(x^*)=0$),收敛速度降为线性.
    \item 重根处理:使用修正公式恢复二阶收敛:
    \[x_{n+1} = x_n - \frac{F(x_n)}{F'(x_n)}, F(x) = \frac{f(x)}{f'(x)}\]
    \item 注意以上定理并没有提供全局收敛性判定,全局收敛性需要针对具体问题分析。
  \end{itemize}
  \item Aitken加速法:一种后处理技术,可用于加速任何线性收敛序列。
  \begin{itemize}
    \item 对于线性收敛序列$\{x_n\}$,构造新序列$\{\hat{x}_n\}$:
\[
\hat{x}_n = x_n - \frac{(x_{n+1} - x_n)^2}{x_{n+2} - 2x_{n+1} + x_n} = x_n - \frac{\Delta x_n^2}{\Delta^2 x_n}.
\]
  \end{itemize}
  \item 多元牛顿法(Multi-dimensional Newton’s method):非线性方程组求根方法
  \begin{itemize}
    \item 迭代公式
\[
\mathbf{x}^{(m+1)} = \mathbf{x}^{(m)} - [J(\mathbf{x}^{(m)})]^{-1} \mathbf{f}(\mathbf{x}^{(m)})
\]
其中$J$是雅可比矩阵。
\item 实际操作:避免直接求逆,通过求解线性方程组
$$
J(\mathbf{x}^{(m)}) \mathbf{y}^{(m)} = \mathbf{f}(\mathbf{x}^{(m)}),
$$
然后令 $\mathbf{x}^{(m+1)} = \mathbf{x}^{(m)} - \mathbf{y}^{(m)}$.
  \end{itemize}
\end{itemize}

\section{Interpolation: approximation of curves by polynomials}

\begin{itemize}
  \item 基本问题设定
\begin{itemize}
    \item 已知 $n+1$ 个数据点 $(x_i, f(x_i))$,$i=0,1,\ldots,n$(其中 $x_i$ 互不相同)。
    \item 目标是构造一个次数不超过 $n$ 的多项式 $P_n(x)$,使得:
    \[
    P_n(x_i) = f(x_i), \quad i=0,1,\ldots,n
    \]
    这样的 $P_n(x)$ 称为 \textbf{插值多项式}。
    \item 存在性与唯一性:插值多项式存在且唯一。理由:$n+1$ 个点确定一个 $n$ 次多项式的系数。
    可以用线性代数方法证明 (Vandermonde 矩阵的非奇异性)。
\end{itemize}
\item 拉格朗日插值多项式
\begin{itemize}
    \item 拉格朗日基:
    \[
    L_{n,k}(x) = \prod_{\substack{i=0 \\ i \neq k}}^{n} \frac{x - x_i}{x_k - x_i}, \quad k=0,1,\ldots,n
    \]
    性质:$L_{n,k}(x_j) = \delta_{jk}$(克罗内克 $\delta$ 函数)。
    \item Lagrange 插值多项式:
    \[
    P_n(x) = \sum_{k=0}^{n} f(x_k) L_{n,k}(x)
    \]
    \item 插值误差定理:
    设 $f \in C^{n+1}[a,b]$,$x_0, x_1, \ldots, x_n \in [a,b]$ 互异,则对任意 $x \in [a,b]$,存在 $\xi(x) \in (a,b)$ 使得:
    \[
    f(x) - P_n(x) = \frac{f^{(n+1)}(\xi(x))}{(n+1)!} \prod_{i=0}^{n} (x - x_i)
    \]
    这是一个高阶的中值定理。
    \item 给定函数和节点后,可计算误差上界:
    \[
    |f(x) - P_n(x)| \leq \frac{M_{n+1}}{(n+1)!} \max_{x \in [a,b]} \left| \prod_{i=0}^{n} (x - x_i) \right|
    \]
    其中 $M_{n+1} = \max_{\xi \in [a,b]} |f^{(n+1)}(\xi)|$. 讲义中 $f(x)=1/x$ 在 $[1,2]$ 上的二次插值误差分析。
\end{itemize}
\end{itemize}

\section{Differentiation}

\begin{itemize}
  \item 基本思路:
\begin{itemize}
    \item 利用差商近似导数:$ f'(x) \approx \dfrac{f(x+h) - f(x)}{h} $ (或更高阶差商).
    \item 等价方法:先用插值多项式近似函数,再对多项式求导。
    \item 误差估计:
    $$
    f'(x) - \frac{f(x+h) - f(x)}{h} = -\frac{h}{2} f''(\xi), \quad \xi \in (x, x+h).
    $$
    和初值问题的局部截断误差类似。
\end{itemize}
  \item 基本差商公式及误差 (可通过 Taylor 展开推导)
  \begin{table}[h]
\centering
\caption{一阶导数差商公式}
\begin{tabular}{@{}llll@{}}
\hline
\textbf{方法} & \textbf{公式} & \textbf{截断误差} & \textbf{收敛阶} \\ 
\textbf{向前差商} & $ f'(x_0) \approx \dfrac{f(x_0+h)-f(x_0)}{h} $ & $ -\dfrac{h}{2}f''(\xi) $ & $ O(h) $ \\
\textbf{向后差商} & $ f'(x_0) \approx \dfrac{f(x_0)-f(x_0-h)}{h} $ & $ \dfrac{h}{2}f''(\xi) $ & $ O(h) $ \\
\textbf{中心差商} & $ f'(x_0) \approx \dfrac{f(x_0+h)-f(x_0-h)}{2h} $ & $ -\dfrac{h^2}{6}f'''(\xi) $ & $ O(h^2) $ \\ 
\end{tabular}
\end{table}
  \item 二阶导数近似
  \begin{itemize}
    \item \textbf{中心差商公式}:
    \[
    f''(x_0) \approx \frac{f(x_0+h) - 2f(x_0) + f(x_0-h)}{h^2}
    \]
    \item \textbf{截断误差}:$ -\dfrac{h^2}{12}f^{(iv)}(\xi) $,收敛阶 $ O(h^2) $.
\end{itemize}
边值问题时常用此公式离散化二阶导数。
  \item 总误差与最优步长
  \begin{itemize}
    \item \textbf{总误差 = 截断误差 + 舍入误差}, 减小 $ h $ $\rightarrow$ 截断误差减小。减小 $ h $ $\rightarrow$ 舍入误差增大。
    \item 存在\textbf{最优步长} $ h_{\text{opt}} $ 使总误差最小。
    \item \textbf{示例分析}:对于中心差商近似 $ f'(x_0) $
    \[
    \text{总误差} \lesssim \frac{\epsilon|f(x_0)|}{h} + \frac{h^2}{6}|f'''(x_0)|
    \]
    最小化得:
    \[
    h_{\text{opt}} = \left( \frac{3\epsilon |f(x_0)|}{|f'''(x_0)|} \right)^{1/3}
    \]
    其中 $\epsilon$ 是机器精度。
\end{itemize}
  \item 提高精度的方法:Richardson外推,外推过程示例:
对于中心差商公式,已知:
\[
f'(x_0) = \frac{f(x_0+h)-f(x_0-h)}{2h} + a_2 h^2 + a_4 h^4 + \cdots
\]

\textbf{第一步外推}(消除 $ h^2 $ 项):
\[
f'(x_0) \approx \frac{4D(h/2) - D(h)}{3} + O(h^4)
\]
其中 $ D(h) = \dfrac{f(x_0+h)-f(x_0-h)}{2h} $.

\textbf{第二步外推}(消除 $ h^4 $ 项):
\[
f'(x_0) \approx \frac{16D^{(1)}(h/2) - D^{(1)}(h)}{15} + O(h^6)
\]
其中 $ D^{(1)}(h) $ 是第一步外推结果.

任何带有已知阶截断误差 $O(h^p)$ 的数值方法均可用此法提高精度。将这个思想用到数值积分中,就是 Romberg 积分法。
\end{itemize}

\section{Integration}

\begin{itemize}
\item 数值积分(求积)的基本形式
\[
I = \int_a^b f(x) \, dx \approx \sum_{i=1}^n c_i f(x_i)
\]
$x_i$:求积节点;$c_i$:求积系数(权重);

\item 代数精度
\begin{itemize}
    \item 若求积公式对任意次数 $\leq m$ 的多项式精确成立,则称其具有 \textbf{$m$次代数精度}
    \item 代数精度越高,对光滑函数逼近越好
\end{itemize}

\item 基本求积公式及其误差
\begin{itemize}
\item 梯形公式
\[
\int_{x_0}^{x_1} f(x) dx = \frac{h}{2} [f(x_0) + f(x_1)] - \frac{h^3}{12} f''(\xi), \quad h = x_1 - x_0
\]
几何意义:用梯形面积近似曲边梯形面积;代数精度:1次(对线性函数精确)
\item Simpson公式
\[
\int_{x_0}^{x_2} f(x) dx = \frac{h}{3} [f(x_0) + 4f(x_1) + f(x_2)] - \frac{h^5}{90} f^{(iv)}(\xi), \quad h = \frac{x_2 - x_0}{2}
\]
虽然基于二次插值,但对三次多项式精确(代数精度3次);误差阶 $O(h^5)$.
\end{itemize}

\item 复合求积公式:将区间细分为多个等距小区间,在每个小区间上使用低阶求积公式,然后累加结果。
\begin{itemize}
  \item 复合梯形公式
\[
T_n = \frac{h}{2} \left[ f(a) + 2\sum_{i=1}^{n-1} f(x_i) + f(b) \right], \quad h = \frac{b-a}{n}
\]
误差:$-\dfrac{h^2}{12}(b-a) f''(\xi)$,$O(h^2)$.
\item 复合Simpson公式($n$为偶数)
\[
S_n = \frac{h}{3} \left[ f(a) + 4\sum_{\substack{i=1 \\ i \text{ odd}}}^{n-1} f(x_i) + 2\sum_{\substack{i=2 \\ i \text{ even}}}^{n-2} f(x_i) + f(b) \right]
\]
误差:$-\dfrac{h^4}{180}(b-a) f^{(iv)}(\xi)$,$O(h^4)$.
\end{itemize}

\item 加速技术:Romberg积分
\begin{itemize}
  \item 核心思想: 对复合梯形公式进行\textbf{Richardson外推};
  利用梯形公式误差展开为\textbf{偶次幂级数}:
    \[
    I = T_n + a_2 h^2 + a_4 h^4 + a_6 h^6 + \cdots
    \]
  \item 外推过程
  \begin{enumerate}
    \item \textbf{第一次外推}(消去 $T_n$ 的 $h^2$项):
    \[
    T_n^{(1)} = \frac{4T_n - T_{n/2}}{3} = S_n \quad (\text{恰好得到Simpson公式})
    \]
    \item \textbf{第二次外推}(消去 $T_n^{(1)}$ 的误差 $h^4$项):
    \[
    T_n^{(2)} = \frac{16T_n^{(1)} - T_{n/2}^{(1)}}{15}
    \]
    \item 可继续外推得到更高阶公式
  \end{enumerate}
\end{itemize}
\item 正交多项式理论(Gauss求积的基础)
\begin{itemize}
  \item 正交多项式的定义:对于权函数 $w(x) \geq 0$,多项式序列 $\{\phi_n(x)\}$ 满足:
\[
\langle \phi_m, \phi_n \rangle = \int_a^b w(x) \phi_m(x) \phi_n(x) dx = 
\begin{cases}
0, & m \neq n \\
\alpha_n > 0, & m = n
\end{cases}
\]
  \item 重要性质
\begin{enumerate}
    \item $\phi_n(x)$ 在 $(a,b)$ 内有 $n$ 个互异实根;
    \item 低次多项式可用正交多项式线性表示(是一组基);
    \item \textbf{三项递推关系}(Gram-Schmidt过程导出):
    \[
    \phi_{k+1}(x) = A_{k+1} \left( x\phi_k(x) - \frac{\langle x\phi_k, \phi_k \rangle}{\langle \phi_k, \phi_k \rangle} \phi_k(x) - \frac{\langle x\phi_k, \phi_{k-1} \rangle}{\langle \phi_{k-1}, \phi_{k-1} \rangle} \phi_{k-1}(x) \right)
    \]
\end{enumerate}
  \item 经典正交多项式族
\begin{table}[h]
\centering
\caption{经典正交多项式族}
\begin{tabular}{@{}llll@{}}
\hline
\textbf{名称} & \textbf{区间 $[a,b]$} & \textbf{权函数 $w(x)$} & \textbf{标准记号} \\ \hline
Legendre & $[-1, 1]$ & $1$ & $P_n(x)$ \\
Chebyshev (第一类) & $[-1, 1]$ & $(1-x^2)^{-1/2}$ & $T_n(x)$ \\
Laguerre & $[0, \infty)$ & $e^{-x}$ & $L_n(x)$ \\
Hermite & $(-\infty, \infty)$ & $e^{-x^2}$ & $H_n(x)$ \\ \hline
\end{tabular}
\end{table}
\end{itemize}
\end{itemize}

\section{Ordinary differential equations: initial value problems(IVPs)}

\begin{itemize}
  \item 初值问题(IVP)的标准形式
\begin{itemize}
    \item \textbf{一阶方程}:
    \[
    \frac{dy}{dt} = f(t, y), \quad a < t < b, \quad y(a) = \alpha
    \]
    \item \textbf{高阶方程转化}:任何 $n$ 阶 ODE 可转化为一阶方程组
    \[
    \begin{cases}
    u_1' = u_2 \\
    u_2' = u_3 \\
    \vdots \\
    u_n' = f(t, u_1, u_2, \ldots, u_n)
    \end{cases}
    \]
    \item 记 $\mathbf{u}(t) = (u_1, u_2, \ldots, u_n)^T$,则化为向量形式 $\mathbf{u}' = \mathbf{f}(t, \mathbf{u})$
\end{itemize}
\item 数值解法的基本思想:将连续时间离散化为 \textbf{网格点} $t_i = a + ih$,$h$ 为步长,
用 \textbf{差分方程} 近似微分方程,逐步推进求解。
\item 欧拉公式
\begin{itemize}
\item 格式
\[
y_{i+1} = y_i + h f(t_i, y_i), \quad i = 0,1,\ldots,N-1
\]
基于向前差商近似导数 $y'(t_i) \approx \frac{y(t_{i+1})-y(t_i)}{h}$.
 \item 局部截断误差定义(使用 Taylor 展开):$\tau_{i+1} = \frac{h^2}{2} y''(\xi_i)$,$O(h^2)$;
 \item 全局误差定义(求解差分方程):$O(h)$ $\rightarrow$ \textbf{一阶方法}。
\end{itemize}
\item 高阶泰勒方法
\begin{itemize}
    \item 利用泰勒展开保留更多项:
    \[
    y_{i+1} = y_i + h y_i' + \frac{h^2}{2} y_i'' + \frac{h^3}{6} y_i''' + \cdots
    \]
    需要计算高阶导数,复杂且实际计算困难。
    \item 二阶泰勒方法示例:
    \[
    y_{i+1} = y_i + h f(t_i, y_i) + \frac{h^2}{2} \left[ f_t(t_i, y_i) + f_y(t_i, y_i) f(t_i, y_i) \right]
    \]
\end{itemize}
\item Runge-Kutta 方法:对 $f(t_i, y_i)$ 做 Taylor 展开并匹配高阶 Taylor 展开的项,避免计算高阶导数。
\begin{itemize}
  \item RK2(中点法)
\[
\begin{aligned}
k_1 &= f(t_i, y_i) \\
k_2 &= f\left(t_i + \frac{h}{2}, y_i + \frac{h}{2} k_1\right) \\
y_{i+1} &= y_i + h k_2
\end{aligned}
\]
局部误差 $O(h^3)$,全局误差 $O(h^2)$.
\item 经典 RK4(最常用)
\[
\begin{aligned}
k_1 &= f(t_i, y_i) \\
k_2 &= f\left(t_i + \frac{h}{2}, y_i + \frac{h}{2} k_1\right) \\
k_3 &= f\left(t_i + \frac{h}{2}, y_i + \frac{h}{2} k_2\right) \\
k_4 &= f(t_i + h, y_i + h k_3) \\
y_{i+1} &= y_i + \frac{h}{6} (k_1 + 2k_2 + 2k_3 + k_4)
\end{aligned}
\]
局部误差 $O(h^5)$,全局误差 $O(h^4)$.
\end{itemize}
\item 多步法(Multi-step Methods)利用多个前步信息:
    \[
    y_{i+1} = \sum_{j=1}^k \alpha_j y_{i+1-j} + h \sum_{j=0}^k \beta_j f_{i+1-j}
    \]
    其中 $f_j = f(t_j, y_j)$, \textbf{显式}:$\beta_0 = 0$,\textbf{隐式}:$\beta_0 \neq 0$.
    通常 $k$ 步法阶数可达到 $k$(显式)或 $k+1$(隐式)。

\item 数值方法的稳定性理论
\begin{itemize}
  \item 零稳定性(Zero-Stability):\newline
  定义:当 $h \to 0$ 时,数值解保持有界;\newline
  判定:特征多项式满足根条件(所有根模 $\leq 1$,模为 $1$ 的根为单根)。
\item 绝对稳定性(A-Stability):\newline
定义:对测试方程 $y' = \lambda y$($\text{Re}(\lambda)<0$),数值解不爆炸(有界)\newline
稳定区域:复平面 $h\lambda$ 的区域,其中数值解衰减。
\end{itemize}
\item 收敛性与误差分析
\begin{itemize}
    \item 收敛性定理:若数值方法需满足:\newline
    一致性:局部截断误差 $\to 0$(阶数 $\geq 1$)\newline
    和零稳定性则全局误差 $O(h^p)$,其中 $p$ 为方法阶数。
\end{itemize}
\end{itemize}

\section{Ordinary differential equations: boundary value problems (BVPs)}

\begin{itemize}
\item 标准二阶线性 BVP
\[
y'' = f(x, y, y') = p(x) y' + q(x) y + r(x), \quad a < x < b
\]
配合可能的如下边界条件:
\begin{itemize}
    \item \textbf{Dirichlet边界条件}:
    \[
    y(a) = \alpha, \quad y(b) = \beta
    \]
    \item \textbf{Neumann边界条件}:指定导数值,如 $y'(a) = \alpha$
    \item \textbf{Robin(混合)边界条件}:混合函数值与导数值,如 $ay(a) + by'(a) = \alpha$
\end{itemize}

\item 有限差分法:将区间 $[a,b]$ 离散为网格:$x_i = a + ih$,$i=0,1,\ldots,n$,$h = \frac{b-a}{n}$,
用\textbf{差商}近似导数,将微分方程转化为\textbf{代数方程组}。
\begin{itemize}
  \item 差商近似公式
\[
\begin{aligned}
y'(x_i) &\approx \frac{y_{i+1} - y_{i-1}}{2h} \quad (\text{中心差分,误差} O(h^2)) \\
y''(x_i) &\approx \frac{y_{i+1} - 2y_i + y_{i-1}}{h^2} \quad (\text{中心差分,误差} O(h^2))
\end{aligned}
\]

\item 离散化过程(以Dirichlet边界为例)
在内部节点 $x_i$($i=1,2,\ldots,n-1$)处:
\[
\frac{y_{i+1} - 2y_i + y_{i-1}}{h^2} = p_i \frac{y_{i+1} - y_{i-1}}{2h} + q_i y_i + r_i
\]
其中 $p_i = p(x_i)$,$q_i = q(x_i)$,$r_i = r(x_i)$,$y_i \approx y(x_i)$.

\item 边界条件处理
\begin{itemize}
    \item \textbf{Dirichlet条件}:直接代入 $y_0 = \alpha$,$y_n = \beta$
    \item \textbf{Neumann条件}:如 $y'(a) = \alpha$,用差商近似:
    \[
    \frac{y_1 - y_0}{h} = \alpha \quad (\text{向前差分})
    \]
    或更精确的:
    \[
    \frac{y_1 - y_{-1}}{2h} = \alpha \quad (\text{引入虚拟节点} y_{-1})
    \]
\end{itemize}

\item 形成的线性方程组
整理后得到\textbf{三对角方程组}:
\[
\begin{bmatrix}
a_1 & c_1 & 0 & \cdots & 0 \\
b_2 & a_2 & c_2 & \ddots & \vdots \\
0 & \ddots & \ddots & \ddots & 0 \\
\vdots & \ddots & b_{n-2} & a_{n-2} & c_{n-2} \\
0 & \cdots & 0 & b_{n-1} & a_{n-1}
\end{bmatrix}
\begin{bmatrix}
y_1 \\ y_2 \\ \vdots \\ y_{n-2} \\ y_{n-1}
\end{bmatrix}
=
\begin{bmatrix}
d_1 \\ d_2 \\ \vdots \\ d_{n-2} \\ d_{n-1}
\end{bmatrix}
\]
其中系数:
\[
\begin{aligned}
a_i &= -2 - h^2 q_i \\
b_i &= 1 + \frac{h}{2} p_i \\
c_i &= 1 - \frac{h}{2} p_i \\
d_1 &= h^2 r_1 - \alpha\left(1 + \frac{h}{2} p_1\right) \\
d_i &= h^2 r_i, \quad i=2,\ldots,n-2 \\
d_{n-1} &= h^2 r_{n-1} - \beta\left(1 - \frac{h}{2} p_{n-1}\right)
\end{aligned}
\]

\end{itemize}

\end{itemize}


% \subsubsection*{6. 求解与误差}
% \begin{itemize}
%     \item \textbf{求解方法}:追赶法(Thomas算法),$O(n)$ 运算量
%     \item \textbf{误差分析}:整体截断误差 $O(h^2)$
% \end{itemize}

% \subsection*{三、谱方法简介(课程省略部分)}

% \subsubsection*{1. 基本思想}
% \begin{itemize}
%     \item 将解展开为\textbf{光滑基函数}的线性组合:
%     \[
%     y(x) \approx \tilde{y}(x) = \sum_{n=1}^N c_n \phi_{n-1}(x)
%     \]
%     \item 基函数通常选择\textbf{正交多项式}(如Chebyshev、Legendre)
% \end{itemize}

% \subsubsection*{2. Galerkin方法}
% \begin{itemize}
%     \item 要求残差 $\mathcal{L}\tilde{y} - r(x)$ 与所有基函数正交:
%     \[
%     \langle \phi_{m-1}, \mathcal{L}\tilde{y} - r \rangle = 0, \quad m=1,\ldots,N-2
%     \]
%     \item 结合边界条件,得到 $N$ 个方程的线性系统
% \end{itemize}

% \subsubsection*{3. 优点与缺点}
% \begin{itemize}
%     \item \textbf{优点}:
%     \begin{enumerate}
%         \item \textbf{指数收敛}:对光滑函数,误差随 $N$ 指数衰减
%         \item \textbf{全局表示}:得到连续函数而不仅是离散点
%     \end{enumerate}
%     \item \textbf{缺点}:
%     \begin{enumerate}
%         \item 实现复杂
%         \item 对非光滑解效果差
%         \item 形成的矩阵可能稠密
%     \end{enumerate}
% \end{itemize}

% \subsubsection*{4. 示例:Chebyshev谱方法}
% \begin{itemize}
%     \item 区间 $[-1,1]$,基函数 $T_n(x) = \cos(n \arccos x)$
%     \item 节点:$x_j = \cos\left(\frac{(2j-1)\pi}{2N}\right)$(Chebyshev-Gauss节点)
%     \item 快速收敛但需要处理端点奇异性
% \end{itemize}

% \subsection*{四、打靶法(Shooting Method)概念}

% \subsubsection*{1. 基本思想}
% \begin{itemize}
%     \item 将BVP转化为\textbf{参数化的IVP}
%     \item 猜测缺失的初始条件,积分到另一端,调整参数满足边界条件
%     \item 相当于求根问题:$F(s) = y(b;s) - \beta = 0$,其中 $s = y'(a)$
% \end{itemize}

% \subsubsection*{2. 算法步骤}
% \begin{enumerate}
%     \item 猜测初始斜率 $s^{(0)}$
%     \item 求解IVP:$y'' = f(x,y,y')$,$y(a)=\alpha$,$y'(a)=s^{(k)}$
%     \item 计算误差:$e^{(k)} = y(b) - \beta$
%     \item 用求根法(如牛顿法)更新 $s^{(k+1)}$
%     \item 重复直到收敛
% \end{enumerate}

% \subsubsection*{3. 优缺点}
% \begin{itemize}
%     \item \textbf{优点}:可利用现有IVP求解器,概念简单
%     \item \textbf{缺点}:可能不稳定,对初值敏感,计算量可能较大
% \end{itemize}

% \subsection*{五、方法比较与选择指南}

% \begin{table}[h]
% \centering
% \caption{边值问题数值方法比较}
% \begin{tabular}{@{}lllll@{}}
% \toprule
% \textbf{方法} & \textbf{核心思想} & \textbf{精度} & \textbf{稳定性} & \textbf{适用场景} \\ \midrule
% \textbf{有限差分法} & 局部差分离散化 & $O(h^2)$ & 稳定 & 一般线性/非线性问题,边界规则 \\
% \textbf{谱方法} & 全局基函数展开 & 指数收敛(光滑解) & 可能不稳定 & 非常光滑的解,周期性边界 \\
% \textbf{打靶法} & BVP $\to$ IVP + 求根 & 依赖IVP求解器 & 可能不稳定 & 简单边界,好的初值猜测 \\ \bottomrule
% \end{tabular}
% \end{table}

% \subsection*{六、实际计算建议}

% \subsubsection*{1. 线性问题}
% \begin{itemize}
%     \item 有限差分法 + 三对角求解器是最佳选择
%     \item 可考虑高阶差分格式提高精度
% \end{itemize}

% \subsubsection*{2. 非线性问题}
% \begin{itemize}
%     \item \textbf{有限差分}:离散后得非线性方程组,用牛顿迭代
%     \item \textbf{打靶法}:结合IVP求解器和求根算法
%     \item \textbf{拟线性化}:迭代求解线性化问题
% \end{itemize}

% \subsubsection*{3. 边界条件处理技巧}
% \begin{itemize}
%     \item \textbf{Neumann条件}:用中心差分避免精度损失
%     \item \textbf{混合条件}:类似处理,保持对称性
%     \item \textbf{周期性边界}:$y_0 = y_n$,修改矩阵结构
% \end{itemize}

% \subsubsection*{4. 网格细化策略}
% \begin{itemize}
%     \item 均匀网格:简单但可能效率低
%     \item 自适应网格:在解变化快处加密(基于误差估计)
% \end{itemize}

% \subsubsection*{5. 验证与误差估计}
% \begin{itemize}
%     \item \textbf{收敛性测试}:减半步长,观察误差变化
%     \item \textbf{Richardson外推}:提高精度估计
%     \item \textbf{解析解比较}:如有简单特例
% \end{itemize}

% \subsection*{七、关键思想总结}

% \subsubsection*{1. 从局部到全局的视角转换}
% \begin{itemize}
%     \item IVP:局部推进,关注时间演化
%     \item BVP:全局耦合,关注空间分布
% \end{itemize}

% \subsubsection*{2. 离散化策略的选择}
% \begin{itemize}
%     \item \textbf{有限差分}:物理空间离散,直观但精度有限
%     \item \textbf{谱方法}:频率空间表示,高效但对光滑性敏感
% \end{itemize}

% \subsubsection*{3. 边界条件的核心作用}
% \begin{itemize}
%     \item 不仅影响解的唯一性,还影响数值方法的构造
%     \item 不同类型的边界需要不同的离散策略
% \end{itemize}

% \subsubsection*{4. 线性与非线性问题的区别}
% \begin{itemize}
%     \item 线性:直接求解线性系统
%     \item 非线性:需要迭代(牛顿法、拟线性化)
% \end{itemize}

% \subsubsection*{5. 打靶法的哲学}
% \begin{itemize}
%     \item 将空间边界问题转化为时间初值问题
%     \item 体现了问题转化的数学思想
% \end{itemize}

% \subsection*{八、与其他章节的联系}

% \subsubsection*{1. 与线性代数的关系}
% \begin{itemize}
%     \item 有限差分法产生稀疏矩阵(三对角)
%     \item 谱方法可能产生稠密但结构化的矩阵
% \end{itemize}

% \subsubsection*{2. 与数值积分的关系}
% \begin{itemize}
%     \item 某些积分方程可转化为微分方程边值问题
% \end{itemize}

% \subsubsection*{3. 与偏微分方程的关系}
% \begin{itemize}
%     \item BVP数值方法是PDE数值方法的基础(如椭圆型方程)
% \end{itemize}

% \subsubsection*{4. 与求根方法的关系}
% \begin{itemize}
%     \item 打靶法需要求解非线性方程(牛顿法等)
% \end{itemize}

% \subsection*{实用建议}
% \begin{enumerate}
%     \item \textbf{首选有限差分法}:除非有特殊理由(如周期性、高光滑性)
%     \item \textbf{测试多个网格}:验证收敛性
%     \item \textbf{检查边界处理}:边界误差常常主导整体误差
%     \item \textbf{利用对称性}:如有,可减少计算量
%     \item \textbf{考虑物理背景}:选择合适的离散化方法
% \end{enumerate}

% \textbf{边值问题的数值求解在工程和科学中极为常见}(如结构力学、量子力学、热传导等)。掌握有限差分法这一基础工具,并了解其他高级方法的思想,是解决实际工程问题的关键能力。

% \end{document}



% \subsubsection*{2. 误差来源}
% \begin{itemize}
%     \item \textbf{局部截断误差}:单步误差
%     \item \textbf{传播误差}:前面误差的放大
%     \item \textbf{舍入误差}:计算机有限精度
% \end{itemize}

% \subsection*{七、刚性问题(Stiff Problems)}

% \subsubsection*{1. 刚性现象}
% \begin{itemize}
%     \item 方程包含快变和慢变分量
%     \item 显式方法需要极小的步长(由快变分量决定)
%     \item 示例:$y' = -100(y - \cos t) - \sin t$,解 $y(t)=\cos t$
% \end{itemize}

% \subsubsection*{2. 刚性比}
% \[
% R = \frac{\max |\text{Re}(\lambda)|}{\min |\text{Re}(\lambda)|}
% \]
% \begin{itemize}
%     \item $R$ 很大时问题刚性严重
% \end{itemize}

% \subsubsection*{3. 求解策略}
% \begin{itemize}
%     \item 使用 \textbf{隐式方法}(A-稳定)
%     \item 常用方法:隐式欧拉、梯形法、BDF、Rosenbrock 方法
%     \item 避免使用显式 Runge-Kutta
% \end{itemize}

% \subsection*{八、方法比较与选择指南}

% \begin{table}[h]
% \centering
% \caption{常微分方程数值方法比较}
% \begin{tabular}{@{}lllll@{}}
% \toprule
% \textbf{方法} & \textbf{类型} & \textbf{阶数} & \textbf{稳定性} & \textbf{适用场景} \\ \midrule
% \textbf{欧拉法} & 显式 & 1 & 条件稳定 & 教学,简单问题 \\
% \textbf{隐式欧拉} & 隐式 & 1 & A-稳定 & 刚性问题 \\
% \textbf{梯形法} & 隐式 & 2 & A-稳定 & 一般问题,中等精度 \\
% \textbf{RK4} & 显式 & 4 & 条件稳定 & 一般问题,高精度 \\
% \textbf{AB4} & 显式 & 4 & 条件稳定 & 光滑问题,高效率 \\
% \textbf{BDF2} & 隐式 & 2 & 刚稳定 & 刚性问题 \\ \bottomrule
% \end{tabular}
% \end{table}

% \subsubsection*{计算成本备注:}
% \begin{itemize}
%     \item \textbf{显式方法}:每步计算一次或几次函数值 $f$
%     \item \textbf{隐式方法}:每步需解方程(可能非线性),通常用牛顿迭代
%     \item \textbf{多步法}:启动后每步计算一次 $f$(高效)
% \end{itemize}

% \subsection*{九、实际计算建议}

% \subsubsection*{1. 步长选择}
% \begin{itemize}
%     \item \textbf{固定步长}:适用于光滑、非刚性、变化缓慢的问题
%     \item \textbf{自适应步长}:基于误差估计自动调整步长
%     \begin{itemize}
%         \item 嵌入对:如 RK4(5) $\rightarrow$ Dormand-Prince 方法
%         \item 误差估计:比较不同阶方法的结果
%     \end{itemize}
% \end{itemize}

% \subsubsection*{2. 误差控制}
% \begin{itemize}
%     \item \textbf{局部误差估计}:比较不同阶方法的结果
%     \item \textbf{事后误差估计}:Richardson 外推
% \end{itemize}

% \subsubsection*{3. 启动策略}
% \begin{itemize}
%     \item 多步法:用单步法(如 RK)计算前几步
%     \item 从大步长开始,逐步细化
% \end{itemize}

% \subsubsection*{4. 方程组求解}
% \begin{itemize}
%     \item 所有方法直接推广到向量形式
%     \item 隐式方法需要解非线性方程组:牛顿迭代
% \end{itemize}

% \subsection*{十、与其他章节的联系}

% \subsubsection*{1. 与数值微分的关系}
% \begin{itemize}
%     \item 欧拉法基于向前差商
%     \item 高阶方法基于高阶差商或泰勒展开
% \end{itemize}

% \subsubsection*{2. 与数值积分的关系}
% \begin{itemize}
%     \item 隐式方法对应于积分近似
%     \begin{itemize}
%         \item 梯形法 $\leftrightarrow$ 梯形积分
%         \item Adams 方法 $\leftrightarrow$ 多项式积分
%     \end{itemize}
% \end{itemize}

% \subsubsection*{3. 与线性代数的关系}
% \begin{itemize}
%     \item 求解线性化方程组(隐式方法)
%     \item 稳定性分析涉及特征值
% \end{itemize}

% \subsection*{十一、关键思想总结}

% \subsubsection*{1. 精度与稳定的权衡}
% \begin{itemize}
%     \item 高阶方法精度高,但可能稳定性差
%     \item 隐式方法稳定,但需解方程,计算量大
% \end{itemize}

% \subsubsection*{2. 显式 vs. 隐式}
% \begin{itemize}
%     \item \textbf{显式}:简单、快速,适合非刚性
%     \item \textbf{隐式}:稳定、适合刚性,但复杂
% \end{itemize}

% \subsubsection*{3. 单步 vs. 多步}
% \begin{itemize}
%     \item \textbf{单步}:自启动、易变步长
%     \item \textbf{多步}:高效(函数求值少),但启动复杂
% \end{itemize}

% \subsubsection*{4. 适应性是关键}
% \begin{itemize}
%     \item 没有万能方法
%     \item 根据问题特性(刚性、光滑度、精度要求)选择方法
% \end{itemize}

% \subsubsection*{5. 刚性问题的挑战}
% \begin{itemize}
%     \item 显式方法失效
%     \item 需要专门方法(隐式、指数积分器等)
% \end{itemize}

% \subsection*{实用建议}
% \begin{enumerate}
%     \item \textbf{非刚性问题}:RK4 或 Adams 方法
%     \item \textbf{刚性问题}:隐式方法(BDF、隐式 RK)
%     \item \textbf{未知特性问题}:从 RK4 开始,观察数值行为
%     \item \textbf{高精度需求}:高阶方法 + 自适应步长
%     \item \textbf{大规模问题}:考虑计算效率,可能选多步法
% \end{enumerate}

% \textbf{本章建立的 ODE 数值解法框架是科学计算的基石},广泛应用于物理、生物、化学、工程等领域的动态系统模拟。理解不同方法的稳定性、精度和适用范围,对于正确选择数值方法和获得可靠结果至关重要。








% \subsection*{六、Gauss求积:最优代数精度}

% \subsubsection*{1. 核心定理}
% \begin{itemize}
%     \item 对于权函数 $w(x)$ 和相应的正交多项式 $\phi_n(x)$
%     \item 取 $\phi_n(x)$ 的 $n$ 个零点作为求积节点
%     \item 构造的 $n$ 点求积公式具有 \textbf{$2n-1$ 次代数精度}(达到理论最优)
% \end{itemize}

% \subsubsection*{2. Gauss-Legendre求积(最常用)}
% \begin{itemize}
%     \item 区间 $[-1, 1]$,权函数 $w(x)=1$
%     \item \textbf{节点}:Legendre多项式 $P_n(x)$ 的零点
%     \item \textbf{权重}:通过积分基函数得到
% \end{itemize}

% \paragraph{低阶公式示例:}
% \begin{itemize}
%     \item \textbf{1点公式}(中点公式):$\int_{-1}^1 f(x)dx \approx 2f(0)$,精度1次
%     \item \textbf{2点公式}:$\int_{-1}^1 f(x)dx \approx f(-1/\sqrt{3}) + f(1/\sqrt{3})$,精度3次
%     \item \textbf{3点公式}:$\int_{-1}^1 f(x)dx \approx \frac{5}{9}f(-\sqrt{3/5}) + \frac{8}{9}f(0) + \frac{5}{9}f(\sqrt{3/5})$,精度5次
% \end{itemize}

% \subsubsection*{3. 区间变换}
% 对于一般区间 $[a,b]$,作线性变换:
% \[
% x = \frac{b-a}{2}t + \frac{b+a}{2}, \quad dx = \frac{b-a}{2}dt
% \]
% 将积分化为 $[-1,1]$ 上的标准形式。

% \subsubsection*{4. Gauss-Chebyshev求积的特色}
% \begin{itemize}
%     \item 节点:$x_j = \cos\left(\dfrac{(2j-1)\pi}{2n}\right)$
%     \item 权重:$w_j = \dfrac{\pi}{n}$(全部相等!)
%     \item 显式公式,计算特别方便
% \end{itemize}

% \subsection*{七、特殊积分问题的处理}

% \subsubsection*{1. 常见困难类型}
% \begin{enumerate}
%     \item \textbf{被积函数有奇点}(如 $f(x) = e^{-x}/\sqrt{x}$)
%     \item \textbf{积分区间无限}
%     \item \textbf{被积函数剧烈振荡}
%     \item \textbf{被积函数导数不连续}
% \end{enumerate}

% \subsubsection*{2. 处理技巧}
% \begin{enumerate}
%     \item \textbf{变量替换消去奇点}(例:令 $x=t^2$ 消去 $1/\sqrt{x}$)
%     \item \textbf{区间分割}:在奇点附近细分
%     \item \textbf{映射无穷区间到有限区间}(如 $t=1/x$)
%     \item \textbf{专门振荡积分方法}(不在本章范围)
% \end{enumerate}

% \subsubsection*{3. 示例:$\displaystyle\int_0^\infty \frac{\ln x}{1+x^4} dx$ 的处理}
% \begin{itemize}
%     \item 分解为 $[0,1]$ 和 $[1,\infty)$ 两部分
%     \item 对后者作变换 $x=1/t$
%     \item 合并化简为有限区间上的光滑积分
% \end{itemize}

% \subsection*{八、方法比较与选择指南}

% \begin{table}[h]
% \centering
% \caption{数值积分方法比较}
% \begin{tabular}{@{}lllll@{}}
% \toprule
% \textbf{方法} & \textbf{节点要求} & \textbf{代数精度} & \textbf{优点} & \textbf{缺点} \\ \midrule
% \textbf{复合梯形} & 等距 & 1次 & 简单稳定,易于实现 & 收敛慢 \\[4pt]
% \textbf{复合Simpson} & 等距,偶数子区间 & 3次 & 精度较高,实现简单 & 需要偶数子区间 \\[4pt]
% \textbf{Romberg积分} & 等距,$n=2^k$ & 可高阶 & 精度高,重复利用函数值 & 需要规则网格 \\[4pt]
% \textbf{Gauss求积} & 正交多项式零点 & $2n-1$次 & 精度最高,节点数最少 & 节点不规则,权重需查表 \\[4pt]
% \textbf{Gauss-Chebyshev} & 余弦分布点 & $2n-1$次 & 权重相等,公式简单 & 只适用特定权函数 \\ \bottomrule
% \end{tabular}
% \end{table}

% \subsubsection*{适用场景建议:}
% \begin{itemize}
%     \item \textbf{复合梯形}:快速粗略估计,光滑性差函数
%     \item \textbf{复合Simpson}:一般精度要求,函数较光滑
%     \item \textbf{Romberg积分}:高精度需求,函数光滑
%     \item \textbf{Gauss求积}:最高精度需求,函数非常光滑
%     \item \textbf{Gauss-Chebyshev}:含权 $(1-x^2)^{-1/2}$ 的积分
% \end{itemize}

% \subsection*{九、关键思想与实用建议}

% \subsubsection*{1. 核心思想演进}
% \begin{enumerate}
%     \item \textbf{插值型求积}:用多项式逼近被积函数 $\rightarrow$ 基本公式
%     \item \textbf{分段低次}:避免高次振荡 $\rightarrow$ 复合公式
%     \item \textbf{误差外推}:利用误差展开式 $\rightarrow$ Romberg积分
%     \item \textbf{节点优化}:选择节点最大化精度 $\rightarrow$ Gauss求积
% \end{enumerate}

% \subsubsection*{2. 收敛阶与计算成本}
% \begin{itemize}
%     \item 代数精度 $\neq$ 实际误差:还依赖函数的光滑性
%     \item \textbf{复合公式}:误差 $O(h^p)$,成本 $O(n)$
%     \item \textbf{Gauss公式}:误差 $O(C^{-n})$(指数收敛!),但函数需非常光滑
% \end{itemize}

% \subsubsection*{3. 稳定性考虑}
% \begin{itemize}
%     \item 正系数的公式更稳定(梯形、Simpson、Gauss)
%     \item 高次Newton-Cotes可能出现负系数 $\rightarrow$ 数值不稳定
% \end{itemize}

% \subsubsection*{4. 自适应策略}
% 实际软件常采用:
% \begin{enumerate}
%     \item 先用低阶公式估算
%     \item 估计误差,在误差大处细分
%     \item 递归进行直到满足精度
% \end{enumerate}

% \subsection*{十、与其他章节的联系}

% \subsubsection*{1. 与插值的关系}
% \begin{itemize}
%     \item 所有插值型求积公式都基于\textbf{用插值多项式代替被积函数}
%     \item 误差分析直接来自插值误差定理
% \end{itemize}

% \subsubsection*{2. 与微分方程的关系}
% \begin{itemize}
%     \item 常微分方程数值解中,积分公式用于近似时间积分
%     \item 例如:隐式方法中的积分近似
% \end{itemize}

% \subsubsection*{3. 与函数逼近的关系}
% \begin{itemize}
%     \item Gauss求积基于正交多项式理论
%     \item 体现了用特殊函数系展开的思想
% \end{itemize}

% \subsection*{总结要点}
% \begin{enumerate}
%     \item \textbf{没有绝对最好的方法}:需权衡精度需求、函数性质、计算成本
%     \item \textbf{Gauss求积是精度冠军}:但对函数光滑性要求高,节点不规则
%     \item \textbf{Romberg是实用之选}:在规则网格上实现高精度,易于编程
%     \item \textbf{注意异常积分}:奇点、无穷区间等需要特殊处理
%     \item \textbf{误差估计很重要}:实际计算中应监控误差,必要时自适应细分
% \end{enumerate}

% \textbf{数值积分是应用最广泛的数值方法之一},本章建立的理论框架和实用方法为科学计算奠定了坚实基础。理解不同方法背后的思想(而不仅是公式),才能在实际问题中选择合适的积分策略。

\bibliography{../bibtexlib/crazyfish.bib}
\bibliographystyle{plain}

\end{document}

