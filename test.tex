%!Tex Program = xelatex
%\documentclass[a4paper]{article}
\documentclass[a4paper]{ctexart}
\usepackage{xltxtra}
%\setmainfont[Mapping=tex-text]{AR PL UMing CN:style=Light}
%\setmainfont[Mapping=tex-text]{AR PL UKai CN:style=Book}
%\setmainfont[Mapping=tex-text]{WenQuanYi Zen Hei:style=Regular}
%\setmainfont[Mapping=tex-text]{WenQuanYi Zen Hei Sharp:style=Regular}
%\setmainfont[Mapping=tex-text]{AR PL KaitiM GB:style=Regular} 
%\setmainfont[Mapping=tex-text]{AR PL SungtiL GB:style=Regular} 
%\setmainfont[Mapping=tex-text]{WenQuanYi Zen Hei Mono:style=Regular} 

\usepackage{listings}
\usepackage{xcolor}
\usepackage{amsmath}
\usepackage{amsthm}
\usepackage{amssymb}
\usepackage{mathrsfs}
\usepackage{enumitem}  
\usepackage{tikz}

\definecolor{codegreen}{rgb}{0,0.6,0}
\definecolor{codegray}{rgb}{0.5,0.5,0.5}
\definecolor{codepurple}{rgb}{0.58,0,0.82}
\definecolor{backcolour}{rgb}{0.95,0.95,0.92}

\lstdefinestyle{mystyle}{
    backgroundcolor=\color{backcolour},   
    commentstyle=\color{codegreen},
    keywordstyle=\color{magenta},
    numberstyle=\tiny\color{codegray},
    stringstyle=\color{codepurple},
    basicstyle=\ttfamily\footnotesize,
    breakatwhitespace=false,         
    breaklines=true,                 
    captionpos=b,                    
    keepspaces=true,                 
    numbers=left,                    
    numbersep=5pt,                  
    showspaces=false,                
    showstringspaces=false,
    showtabs=false,                  
    tabsize=2
}

\lstset{style=mystyle}

\title{数值分析}
\author{王何宇}
\date{}

\newcommand{\remark}[1]
{\noindent {\bf Remark {#1}}}

\newcommand{\definition}[1]
{\noindent {\bf Definition {#1}}}

\newcommand{\hl}[1]
{\noindent {\bf {#1}}}
\begin{document}
\maketitle
\pagestyle{empty}

% 用 Newton 法寻找函数
% $$
% f(x) = \frac{2}{x + 1} - 1
% $$
% 的零点。

% (i) 分别用 $x_0 = 3$ 和 $x_0 = 4$ 计算一步 Newton 迭代。给出具体的计算过程和对应的 $x_1$.
% (ii) 请问上述两个初值是否收敛到函数零点,为什么?
% (iii) 给出收敛初值的范围.
% (iv) 证明结论 (iii). 


\noindent 1. Use the Newton method to find the zero of the function 
\[ 
f(x) = \frac{2}{x+1} - 1. 
\]
\begin{enumerate}
    \item[i] Perform one step of the Newton iteration with \( x_0 = 3 \) 
    and \( x_0 = 4 \). Provide the detailed calculation process 
    and the corresponding \( x_1 \).
    \item[ii] Determine whether the above two initial values 
    converge to the zero of the function, and explain why.
    \item[iii] Give the range of initial values that converge.
    \item[iv] Prove the conclusion in (iii). 
\end{enumerate}


\noindent 2. Given that the function \( f(x) \) satisfies 
the following conditions at points \( x_0 \) and \( x_1 \):
\[ 
    f(x_0) = f_0, f(x_1) = f_1, f'(x_0) = f'_0, f'(x_1) = f'_1 
\]
\begin{enumerate}
    \item[i] Provide a numerical calculation formula for \( f'(x) \) 
    that uses only function values and has a \( p \)-th order error 
    of \( O(h^2) \). (where \( h = x_1 - x_0 \))
    \item[ii] Provide a numerical calculation formula for \( f''(x) \).
    \item[iii] Provide a numerical calculation formula for \( f'''(x) \) 
    and give the \( p \)-th order error estimate. 
    (Hint: Use Hermite interpolation polynomial)
    \item[iv] Provide a numerical calculation formula for 
    \( f^{(4)}(x) \), then give and prove the \( p \)-th order error estimate.
\end{enumerate}


\noindent 3. Given the parametric equations \( (x(t), y(t)) \), 
\( t \in [0, 2\pi] \) to draw a unit circle, 
passing through the known points \( (1,0) \), \( (0,1) \), \( (-1,0) \), 
and \( (0,-1) \).

\begin{enumerate}
    \item[i] Approximate \( x(t) \) and \( y(t) \) 
    using Lagrange interpolation polynomials: Take
    \[
    t_{i+1} = t_i + \frac{\pi}{2}, \quad t_0 = 0, \quad i = 1, 2, 3, 4.
    \]
    Construct Lagrange interpolation polynomials \( P_x(t) \) and \( P_y(t) \) 
    through \( (t_i, x(t_i)) \) and \( (t_i, y(t_i)) \), respectively. 
    Please write out the expression for \( P_x(t) \).
    \item[ii] Please write out the interpolation remainder.
    % Please provide the supremum (least upper bound) for the error of
    % %  Please provide an error estimate for \( P_x(t) \) 
    % % and indicate when \( t \) is such that the error
    % \[
    % e_L := |P_x(t) - x(t)|.
    % \]
    \item[iii] Further approximate the unit circle's parametric equations 
    \( x(t) \) and \( y(t) \) using periodic splines. 
    Let the corresponding cubic spline functions for \( x(t) \) and 
    \( y(t) \) be \( S_x(t) \) and \( S_y(t) \), respectively. 
    Please set \( d_i = S'_y(t_i) \) as unknowns and provide 
    the system of linear equations for solving \( d_i \).
    (Note: It is known that \( S_y(t_i) = y(t_i) \))
    \item[iv] Please indicate when \( t \) is such that the error
    \[
    e_S := \sqrt{(S_x(t) - x(t))^2 + (S_y(t) - y(t))^2}
    \]
    is maximized.
\end{enumerate}

\noindent 4. Please provide the best square approximation polynomial for 
\( f(x) = \sin(x) \) on the interval \([- \frac{\pi}{2}, \frac{\pi}{2}]\):

\begin{itemize}
    \item[i] Please write out the expression 
    for the best linear approximation polynomial \( P_1(x) \).
    \item[ii] Please write out the normal equations 
    for the best quadratic approximation polynomial \( P_2(x) \) of \( f(x) \) 
    in the basis \( \{1, x, x^2\} \).
    \item[iii] Please orthogonalize the basis \( \{1, x, x^2\} \) 
    using the Gram-Schmidt method. Write out the specific steps.
    \item[iv] Find the best quadratic approximation polynomial \( P_2(x) \) 
    of \( f(x) \) using the basis from question iii.
\end{itemize}

\noindent 5. For \( f(x) = \cos(x) \), 
compute the numerical integral over the interval \([0, \pi]\):
\begin{enumerate}
    \item[i] Please provide the formula for calculating the integral 
    of \( f(x) \) using the trapezoidal rule, 
    and compute the integral value.
    \item[ii] Please provide the formula for calculating the integral 
    of \( f(x) \) using Simpson's rule, 
    compute the integral value, and estimate the error.
    \item[iii] It is known that on the interval \([-1, 1]\), 
    the Legendre polynomials are given by:
    \[
       P_0(x) = 1, 
       \quad P_n(x) = \frac{1}{2^n n!} \frac{d^n}{dx^n} \left( (x^2 - 1)^n \right), 
       \quad n = 1, 2, \ldots
    \]
    Please construct the Gauss-Legendre quadrature formula for \( n = 2 \), 
    and compute the integral of \( f(x) \) over \([0, \pi]\). 
    (Hint: Legendre polynomials have the property: \((1 - x^2) P_n'(x) = n[P_{n-1}(x) - xP_n(x)]\).)
    \item[iv] Verify the algebraic accuracy of the formula obtained in question 3.
\end{enumerate}

\noindent 6. Given the improved Euler's method formula as:
\[
y_{n+1} = y_n + \frac{h}{2} [f(x_n, y_n) + f(x_{n+1}, y_{n+1})].
\]
\begin{enumerate}
    \item[i] Please use this format to calculate the initial value problem:
    \[
    \frac{dy}{dx} = y - x, \quad y(0) = 1,
    \]
    and provide the numerical solution for \( y(1) \), taking \( h = 0.1 \).
    \item[ii] Please provide the local truncation error 
    for this format and indicate the order of the method.
    \item[iii] Please calculate the region of absolute stability for this format.
\end{enumerate}

\noindent 7. Given the boundary value problem:
\[
-\Delta u = \pi^2 \sin(\pi x) \sin(\pi y), 
\quad u|_{\partial \Omega} = \sin(\pi x) \sin(\pi y),
\]
where 
\( \Omega = \left[\frac{1}{2}, \frac{3}{2}\right] \times \left[\frac{1}{2}, 
\frac{3}{2}\right] \).
\begin{enumerate}
    \item[i] Please provide the truncation error for this difference scheme.
    \item[ii] Please use the five-point difference scheme 
    to calculate the numerical solution of \( u(x, y) \), taking \( h = 0.25 \). 
    Provide the node numbering and list the system 
    of linear equations without solving.
\end{enumerate}


\noindent 8. There is an old computer with a floating-point system 
\( \mathcal{F} \) with base \( \beta = 2 \), precision \( p = 3 \), 
and exponent range \( [L = -1, U = 2] \).

\begin{enumerate}
    \item[i] What is the largest floating-point number in \( \mathcal{F} \)? 
    What is the positive subnormal floating-point number closest to $0$?
    \item[ii] How many normalized floating-point numbers can be represented 
    in \( \mathcal{F} \)? Write the next floating-point number after $1$.
    \item[iii] In \( \mathcal{F} \), 
    \[
        fl\left(\frac{a}{b}\right) = \frac{a}{b}(1 + \delta), 
        \quad |\delta| < \epsilon_u.
    \]
    Here, \( a \) and \( b \) are floating-point numbers in \( \mathcal{F} \), 
    and \( \epsilon_u \) is the machine precision. 
    Write out the specific value of \( \epsilon_u \).
    \item[iv] In the IEEE system, it is always specified that \( L = 1 - U \). 
    Please analyze the purpose of doing this.
\end{enumerate}

% 8. 现有一台古老的计算机,它的浮点数系统 $\mathcal{F}$ 的基数 $\beta = 2$, 精度 $p = 3$, 
% 指数区间 $[L = -1, U = 2]$.

% \begin{enumerate}
%     \item[i] 请问 $\mathcal{F}$ 的最大浮点数是几?最接近 $0$ 的非规格浮点数是几?
%     \item[ii] 请问 $\mathcal{F}$ 中总共有多少规格化浮点数可以表示? 写出 $1$ 之后的下一个浮点数.
%     \item[iii] 在 $\mathcal{F}$ 中, 
%     $$
%     fl(\frac{a}{b}) = \frac{a}{b}(1 + \delta), |\delta| < \epsilon_u.
%     $$   
%     这里, $a, b$ 是 $\mathcal{F}$ 中的浮点数, $\epsilon_u$ 是机器精度. 写出具体的 $\epsilon_u$ 数值.
% \end{enumerate}
% 已知边值问题
% $$
% -\Delta u = \pi^2\sin \pi x\sin \pi y, 
% \quad u|_{\partial \Omega} = \sin \pi x \sin \pi y,
% $$
% 这里 $\Omega = [\frac{1}{2}, \frac{3}{2}] \times [\frac{1}{2}, \frac{3}{2}]$.

% 1. 请给出该差分格式的截断误差.

% 2. 请用五点差分格式计算 $u(x, y)$ 的数值解, 取 $h = 0.25$. 给出节点编号, 
% 列出线性方程组, 不必计算.

% 已知改进 Euler 法的公式为
% $$
% y_{n+1} = y_n + \frac{h}{2} [f(x_n, y_n) + f(x_{n+1}, y_{n + 1})].
% $$

% 1. 请用该格式计算初值问题:
% $$
% \frac{dy}{dx} = y - x, y(0) = 1,
% $$
% 给出 $y(1)$ 的数值解, 取 $h = 0.1$.

% 2. 请给出该格式的局部截断误差, 并指出这是几阶方法.

% 3. 请计算该格式的绝对稳定域.




% \newpage 
% 对 $f(x) = \cos x$, 在区间 $[0, \pi]$ 上计算数值积分. 

% 1. 请给出用梯形公式计算 $f(x)$ 积分的公式, 并计算积分值.

% 2. 请给出用 Simpson 公式计算 $f(x)$ 积分的公式, 计算积分值, 并估计误差.

% 3. 已知在 $[-1, 1]$ 区间上, Legendre 多项式为:
% $$
% P_0(x) = 1, P_n(x) = \frac{1}{2^n n!} \frac{d^n}{dx^n} (x^2 - 1)^n, 
% n = 1, 2, \ldots.
% $$
% 请构建 $n = 2$ 的 Gauss-Legendre 积分公式, 
% 计算 $f(x)$ 在 $[0, \pi]$ 上的积分值. 
% (提示: Legendre 多项式有性质: $(1 - x^2)P_n'(x) = n[P_{n - 1}(x) - xP_n(x)]$.)

% 4. 验证第3问得到的公式的代数精度.

% 请在区间 $[-\frac{\pi}{2}, \frac{\pi}{2}]$ 上给出 
% $f(x) = \sin(x)$ 的最佳平方逼近多项式: 

% 1. 请写出最佳一次逼近多项式 $P_1(x)$ 的表达式.

% 2. 请在基 $1, x, x^2$ 下写出 $f(x)$ 的最佳二次逼近多项式 $P_2(x)$ 的正规方程组.

% 3. 请用 Gram-Schmidt 正交化方法将 $1, x, x^2$ 正交化. 写出具体步骤.

% 4. 求出 $f(x)$ 在此基下的最佳二次逼近多项式 $P_2(x)$.

% 用参数方程 $(x(t), y(t))$, $t \in [0, 2 \pi]$ 绘制单位圆, 
% 通过已知点 $(1, 0)$, $(0, 1)$, $(-1, 0)$ 和 $(0, -1)$. 

% 1. 用 Lagrange 插值多项式逼近 $x(t)$ 和 $y(t)$: 取
% \[
%     t_{i + 1} = t_i + \frac{\pi}{2}, t_0 = 0, i = 1, 2, 4.
% \]
% 过 $(t_i, x(t_i))$ 和 $(t_i, y(t_i))$ 做 Lagrange 插值多项式 $P_x(t)$ 和 $P_y(t)$, 
% 请写出 $P_x(t)$ 的表达式.

% 2. 请给出 $P_x(t)$ 的误差
% $$
% e_L := |P_x(t) - x(t)|
% $$ 
% 的上界. 

% 3. 再用周期样条逼近单位圆的参数方程 $x(t)$ 和 $y(t)$, 
% 令对应 $x(t)$ 和 $y(t)$ 的三次样条函数分别为 $S_x(t)$ 和 $S_y(t)$, 
% 请以 $d_i = S_y'(t_i)$ 为未知量, 给出求解 $d_i$ 的线性方程组. 

% 4. 请指出 $t$ 等于多少时, 误差
% $$
% e_S := \sqrt{(S_x(t) - x(t))^2 + (S_y(t) - y(t))^2}
% $$
% 最大.


% 已知函数 $f(x)$ 在 $x_0$ 和 $x_1$ 点满足
% \[
% f(x_0) = f_0, f(x_1) = f_1, f'(x_0) = f'_0, f'(x_1) = f'_1.
% \]
% \begin{enumerate}
%     \item[i] 给出 $f'(x)$ 的数值计算公式,要求只使用函数值,且 $p-$ 阶误差为 $O(h^2)$. 
%     ($h = x_1 - x_0$)
%     \item[ii] 给出 $f''(x)$ 的数值计算公式.
%     \item[iii] 给出 $f^{3}(x)$ 的数值计算公式,并给出误差 $p-$ 阶估计.  
%     (提示:使用 Hermite 插值多项式)
%     \item[iv] 给出 $f^{(4)}(x)$ 的数值计算公式,给出并证明误差 $p-$ 阶估计.
% \end{enumerate}

\end{document}

